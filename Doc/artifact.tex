\documentclass{article}
\usepackage[margin=.8in]{geometry}
\usepackage{amsmath}
\usepackage{amsthm,amssymb,proof}
\usepackage{mathpartir,mathabx}
\usepackage{scrextend}
\usepackage{enumitem}
\usepackage{graphicx}
\usepackage{resizegather}
\usepackage{bbold,stmaryrd}
\usepackage{tikz-cd}
\setcounter{section}{1}

\begin{document}
\title{Artifact: A Computational Interpretation of Compact Closed Categories: Reversible Programming with Negative and Fractional Types}
\author{Chao-Hong Chen and Amr Sabry}
\maketitle
\noindent
This document contains a list of claims in the paper and corresponding code.

\section{Core Reversible Language: $\Pi$}

\begin{itemize}
\item The syntax of $\Pi$ in Fig.1 is formalized in \textbf{Pi/Syntax.agda}.
\end{itemize}

\subsection{Abstract Machine Semantics}
\begin{itemize}
\item The $\delta$ function in Fig.2 is defined in \textbf{Pi/Opsem.agda:L47}.
\item The well-formed continuation stacks in Fig.3 is defined in \textbf{Pi/Opsem.agda:L71}.
\item The machine states in Def.1 is defined in \textbf{Pi/Opsem.agda:L81}.
\item The reduction relation in Fig.4 is defined in \textbf{Pi/Opsem.agda:L86}.
\item Lem.2 is proved in \textbf{Pi/NoRepeat.agda:L13}.
\item Lem.3 is proved in \textbf{Pi/NoRepeat.agda:L29}.
\item Def.4 is defined in \textbf{Pi/Eval.agda:L76}.
\item Def.5 is defined in \textbf{Pi/Eval.agda:L121}.
\item Thm.6 is proved in \textbf{Pi/Properties.agda:L31}.
\end{itemize}
\subsection{Interpreter}
\begin{itemize}
\item The interpreter in Fig.5 is defined in \textbf{Pi/Interp.agda:L9}.
\item Thm.7 is proved in \textbf{Pi/Properties.agda:L49}.
\end{itemize}

\section{Termination of Reversible Abstract Machines}
\begin{itemize}
\item The reversible abstract machine in Def.8 is defined in \textbf{RevMachine.agda:L8}.
\item Lem.9 is proved in \textbf{RevNoRepeat.agda:L112}.
\item Lem.10 is proved in \textbf{Pi/Eval.agda:L13}.
\item Thm.11 is proved in \textbf{Pi/Eval.agda:L76}.
\item The reversible abstract machine in Def.12 is defined in \textbf{RevMachine.agda:L15}.
\item Lem.13 is proved in \textbf{PartialRevNoRepeat.agda:L123}.
\end{itemize}

\section{Space and Time Resources and Trade-Offs}
\begin{itemize}
\item $\# \sigma$ is defined in \textbf{TimeSpace.agda:L71}.
\item The examples in the end of this section is in \textbf{TimeSpace.agda:L80-87}.
\end{itemize}

\section{Negative Types: $\Pi^m$}
\subsection{Abstract Machine Semantics}
\begin{itemize}
\item The syntax of $\Pi^m$ is formalized in \textbf{Pi-/Syntax.agda}.
\item Def.14 is defined in \textbf{Pi-/Opsem.agda:L84}.
\item The transition rules in Fig.6 is defined in \textbf{Pi-/Opsem.agda:L91}.
\end{itemize}

\subsection{Properties}
\begin{itemize}
\item Lem.15 is proved in \textbf{Pi-/NoRepeat.agda:L20}.
\item Lem.16 is proved in \textbf{Pi-/NoRepeat.agda:L119}.
\item Lem.17 is proved in \textbf{Pi-/Eval.agda:L23}.
\item $\Pi^m$ is a reversible abstract machine is proved in \textbf{Pi-/NoRepeat.agda:L223}.
\item Def.18 is generalized in Def.20.
\item Def.20 and generalized Thm.19 is in \textbf{Pi-/Eval.agda:L172}. This proof relies on the finitness of execution trace
  for $\Pi^m$ which follows from the finitness of $\Pi^m$ machine states and non-repeating lemma for reversible abstract
  machines (Lem.9).
\item Def.21 is defined in \textbf{Pi-/Eval.agda:L177}.
\item Thm.22 is proved in \textbf{Pi-/Properties.agda:L70}.
\end{itemize}

\subsection{Interpreter}
\begin{itemize}
\item The interpreter is defined in \textbf{Pi-/Interp.agda:L12}.
\item Thm.23 is proved in \textbf{Pi-/Properties.agda:L198}. This proof relies on the finitness of execution trace
  for $\Pi^m$.
\end{itemize}

\subsection{Compact Closed Category}
\begin{itemize}
\item Thm.24 is proved in \textbf{Pi-/Category.agda:L297}. This proof relies on the finitness of execution trace
  for $\Pi^m$.
\item Thm.25 is proved in \textbf{Pi-/Category.agda:L301}. This proof relies on the finitness of execution trace
  for $\Pi^m$.
\item The code for the remark in the end is in \textbf{Pi-/Category.agda:L306-363}.
\end{itemize}


\section{Fractional Types: $\Pi^d$}
\subsection{Abstract Machine Semantics}
\begin{itemize}
\item The syntax of $\Pi^d$ is formalized in \textbf{PiFrac/Syntax.agda}.
\item Def.26 is defined in \textbf{PiFrac/Opsem.agda:L103}.
\item The transition rules in Fig.8 is defined in \textbf{PiFrac/Opsem.agda:L109}.
\end{itemize}

\subsection{Properties}
\begin{itemize}
\item Lem.27 is proved in \textbf{PiFrac/NoRepeat.agda:L17}.
\item Lem.28 is proved in \textbf{PiFrac/NoRepeat.agda:L38}.
\item Lem.29 is proved in \textbf{PiFrac/Eval.agda:L18}.
\item $\Pi^d$ is a partial reversible abstract machine is proved in \textbf{PiFrac/NoRepeat.agda:L113}.
\item Def.30 and Thm.31 is in \textbf{PiFrac/Eval.agda:L96}.
\item Def.32 and Thm.33 is in \textbf{PiFrac/Eval.agda:L153}.
\item Thm.34 is proved in \textbf{PiFrac/Properties.agda:L20}.
\end{itemize}

\subsection{Interpreter}
\begin{itemize}
\item The interpreter is defined in \textbf{PiFrac/Interp.agda:L15}.
\item Thm.35 is proved in \textbf{PiFrac/Properties.agda:L40}.
\end{itemize}

\subsection{Compact Closed Category}
\begin{itemize}
\item Thm.36 is proved in \textbf{PiFrac/Category.agda:L110}.
\end{itemize}

\section{Combining Negative and Fractional Types: $\Pi^{\mathbb{Q}}$}
\subsection{Abstract Machine Semantics}
\begin{itemize}
\item The syntax of $\Pi^{\mathbb{Q}}$ is defined in \textbf{PiQ/Syntax.agda}.
\item Def.37 is defined in \textbf{PiQ/Opsem.agda:L115}.
\item The transition rules in Fig.9 is defined in \textbf{PiQ/Opsem.agda:L123}.
\end{itemize}

\subsection{Properties}
\begin{itemize}
\item Lem.38 is proved in \textbf{PiQ/NoRepeat.agda:L17}.
\item Lem.39 is proved in \textbf{PiQ/NoRepeat.agda:L135}.
\item Lem.40 is proved in \textbf{PiQ/Eval.agda:L24}.
\item $\Pi^{\mathbb{Q}}$ is a partial reversible abstract machine is proved in \textbf{PiQ/NoRepeat.agda:L250}.
\item Def.41 and Thm.42 is in \textbf{PiQ/Eval.agda:L196}. This proof relies on the finitness of execution trace
  for $\Pi^{\mathbb{Q}}$ which follows from the finitness of $\Pi^{\mathbb{Q}}$ machine states and non-repeating lemma for partial
  reversible abstract machines (Lem.13).
\item Def.43 is in \textbf{PiQ/Eval.agda:L201}.
\item Thm.44 is proved in \textbf{PiQ/Properties.agda:L93}.
\end{itemize}

\subsection{Interpreter}
\begin{itemize}
\item The interpreter is defined in \textbf{PiQ/Interp.agda:L20}.
\item The equivalence between interpreter and machine semantics is proved in \textbf{PiQ/Properties.agda:L351}.
  This proof relies on the finitness of execution trace for $\Pi^{\mathbb{Q}}$.
\end{itemize}

\section{Programming with Negative and Fractional Types}
\begin{itemize}
\item All examples except for SAT solver is in \textbf{PiQ/Examples.agda}.
\item The implementation of SAT solver is in \textbf{PiQ/SAT.agda}.
\end{itemize}

\end{document}
